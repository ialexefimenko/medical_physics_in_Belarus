

%!TEX TS-program = xelatex
%!TEX encoding = UTF-8 Unicode

\documentclass[a4paper,10pt]{extarticle}

%\usepackage{blindtext} % Package to generate dummy text throughout this template 


%\usepackage[sc]{mathpazo} % Use the Palatino font
\usepackage[T1]{fontenc} % Use 8-bit encoding that has 256 glyphs
\linespread{1}
\usepackage{microtype} % Slightly tweak font spacing for aesthetics

\usepackage[english,russian]{babel} 
\usepackage{fontspec} 
\defaultfontfeatures{Ligatures={TeX},Renderer=Basic} 
\setmainfont[Ligatures={TeX,Historic}]{Times New Roman} 

% Language hyphenation and typographical rules

\sloppy             % Избавляемся от переполнений
\hyphenpenalty=1000 % Частота переносов
\clubpenalty=10000  % Запрещаем разрыв страницы после первой строки абзаца
\widowpenalty=10000 % Запрещаем разрыв страницы после последней строки абзаца

\usepackage{textcase}    % to connect the registers 


\usepackage{geometry}
\geometry{left=2.5cm}
\geometry{right=2cm}
\geometry{top=2cm}
\geometry{bottom=2.5cm}

%\usepackage[hmarginratio=1:1,top=32mm,columnsep=20pt]{geometry} % Document margins
\usepackage[hang, small,labelfont=bf,up,textfont=it,up]{caption} % Custom captions under/above floats in tables or figures
\usepackage{booktabs} % Horizontal rules in tables

%\usepackage{lettrine} % The lettrine is the first enlarged letter at the beginning of the text

\usepackage{enumitem} % Customized lists
\setlist[itemize]{noitemsep} % Make itemize lists more compact

\usepackage{abstract} % Allows abstract customization
\renewcommand{\abstractnamefont}{\normalfont\bfseries} % Set the "Abstract" text to bold
%\renewcommand{\abstracttextfont}{\normalfont\small\itshape} % Set the abstract itself to small italic text
\renewcommand{\baselinestretch}{1}
\parindent 1.27cm
\usepackage{titlesec} % Allows customization of titles
\renewcommand\thesection{\Roman{section}} % Roman numerals for the sections
\renewcommand\thesubsection{\roman{subsection}} % roman numerals for subsections
\titleformat{\section}[block]{\large\centering\bfseries\MakeTextUppercase}{\thesection.}{1em}{} % Change the look of the section titles
\titleformat{\subsection}[block]{\large}{\thesubsection.}{1em}{} % Change the look of the section titles

\usepackage{fancyhdr} % Headers and footers
\pagestyle{fancy} % All pages have headers and footers
\fancyhead{} % Blank out the default header
\fancyfoot{} % Blank out the default footer
\fancyhead[C]{Ефименко А.Д. $\bullet$ Май 20120 $\bullet$ МГЭМ им. А.Д. Сахарова БГУ} % Custom header text
\fancyfoot[RO,LE]{\thepage} % Custom footer text

\usepackage{titling} % Customizing the title section

\usepackage{hyperref} % For hyperlinks in the PDF



%----------------------------------------------------------------------------------------
%	TITLE SECTION
%----------------------------------------------------------------------------------------

\setlength{\droptitle}{-6\baselineskip} % Move the title up

\pretitle{\begin{center}\bfseries\MakeTextUppercase} % Article title formatting \bfseries -- жирный шрифт \itshape -- курсив
\posttitle{\end{center}} % Article title closing formatting
\title{Медицинская физика в Республике Беларусь} % Article title

\begin{author}
{
\textsc 
{\MakeTextUppercase{\bfseries \normalsize Medical physics in the Republic of Belarus}}\\ \\{\bfseries А.Д. Ефименко} \\ \\ {\bfseries A.D. Ephimenko}\\ \\% Your name
%\thanks{A thank you or further information} \\[1ex] 
\normalsize \itshape Белорусский государственный университет МГЭИ им. А.Д.Сахарова БГУ \\ 
\normalsize \itshape г. Минск Республика Беларусь \\ 
\normalsize \itshape ISEU BSU, Minsk Republic of Belarus \\ % Your institution
\normalsize \href{mailto:ialexefimenko@icloud.com}{ialexefimenko@icloud.com} % Your email address
}
\end{author}
%\date{\today} % Leave empty to omit a date
\renewcommand{\maketitlehookd}
{
\begin{abstract}
\noindent \normalsize \scshape 
% \scshape -- отмена курсива 
 \\


\itshape Keywords: \scshape 
 
\end{abstract}
}


%----------------------------------------------------------------------------------------

\begin{document}

% Print the title
\maketitle

%----------------------------------------------------------------------------------------
%	ARTICLE CONTENTS
%----------------------------------------------------------------------------------------

\section{Введение}

%\lettrine[nindent=0em,lines=3]{L} orem ipsum dolor sit amet, consectetur adipiscing elit.
%\blindtext % Dummy text

%\blindtext % Dummy text

%------------------------------------------------

\section{Общие характеристики специальности}






%------------------------------------------------

\section{Высшее образование и профессиональная подготовка}

\begin{table}
\caption{Пример таблицы}
\centering
\begin{tabular}{llr}
\toprule
\multicolumn{2}{c}{Name} \\
\cmidrule(r){1-2}
First name & Last Name & Grade \\
\midrule
John & Doe & $7.5$ \\
Richard & Miles & $2$ \\
\bottomrule
\end{tabular}
\end{table}

%\blindtext % Dummy text
%
%\begin{equation}
%\label{eq:emc}
%\frac{dp}{dt}=\vec{F}
%\end{equation}

%\blindtext % Dummy text

%------------------------------------------------

\section{Аспекты значимости работников}

\subsection{}

A statement requiring citation \cite{No174}.
%\blindtext % Dummy text

\subsection{}

%\blindtext % Dummy text
%------------------------------------------------

\section{Перспективы развития}

%------------------------------------------------

\section{Рекомендации по продвижению соответствующих специалистов}
%----------------------------------------------------------------------------------------
%	REFERENCE LIST
%----------------------------------------------------------------------------------------

\begin{thebibliography}{99} % Bibliography - this is intentionally simple in this template

\bibitem [1] {No174}
Guibelade E., Christifides S., Caruna C.J., Evans S., van der Putten W.
\newblock Radiation Protection 174. European Guidelines on medical physics expert.  
\newblock Directorate-General for Energy Directorate D — Nuclear Safety. Fuel Cycle Unit D.3 — Radiation Protection.
2014;	%	{\em Human Nature}
% \bibitem [2]
% \bibitem [3]
\end{thebibliography}

%----------------------------------------------------------------------------------------

\end{document}
