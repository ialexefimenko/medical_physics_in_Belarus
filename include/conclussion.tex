
\section{Заключение}

Медицинская физика носит комплексный характер как научной деятельности, она высокозначима с практической точки зрения. Об этом в первую очередь необходимо помнить, когда обсуждаются вопросы по распределению обязанностей медицинского физика в клинических отделениях нашей страны. Ведь медицинский физик -- это специалист с высшим физико-техническим образованием, имеющий диплом университета по физике, математике, вычисли􏰀 тельной технике, механике, электротехнике или электронике и т.д.. Так устанавливает профессию Европейская федерация организаций по медицинской физике. И наш образовательный сектор должен быть нацелен на близкий по обучению уровень профессии. Близкий потому, что одним из основных препятствий для увеличения предложения медицинских физиков в Европе и США является продолжительность времени, необходимого для обучения и подготовки, и в этих причинах нужна бдительность наблюдений и долгосрочного планирования. 

Весь этот сбор и анализ дополнительных данных, распространение и обсуждение полученных результатов, оценка политики, управление рисками, не решает настоящей проблемы -- безответственность в системе образования. Законы функционирования этой системы определяются договоренностями и соглашениями, явными и неявными, между всеми ее образующими. Система образования должна была вроде как приспособиться к произошедшим изменениям, но не выходит. Мы получаем полную оторванность всей системы науки и образования от реальности. Это и есть главная причина несостоятельности профессии медицинский физик в Республике Беларусь. Необходимо инвестировать в перспективу -- практики, стажировки. Опыт ценнее, особенно для этой профессии. 



