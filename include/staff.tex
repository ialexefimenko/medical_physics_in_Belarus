
\section{Кадровая характеристика медицинской физики}
Медицинские физики обеспечивают высокие требования по точности, гарантии качества и безопасности, осуществляют ответственные физико-математические функции, например, по измерениям, обработке и анализу диагностических изображений, дозиметрическому планированию и контролю в процессе лучевого лечения. Медицинские физики совмещают глубокие физико-математические и медицинские знания, непосредственно участвуют в лечебно-диагностическом процессе, разделяют с врачом ответственность за пациента. Основной обязанностью медицинского физика является обеспечение надлежащего физико-технологического уровня и качества повседневных лечебно-диагностических процедур, выполняемых на основе высоких медицинских технологий. В рамках своей компетентности он должен отвечать за стандартизацию и калибровку медико-физического оборудования, точность и безопасность физических методов, используемых в повседневной клинической практике, гарантию качества радиологических лечебно-диагностических процедур, калибровку и метрологическую поверку дозиметрической и радиометрической аппаратуры. И это не все возможности и обязанности работника, получившего звание <<медицинский физик>> \cite{RP174}. В области проведение научных исследований по развитию новых технологий, радиационно-физических методов и техники, медицинский физик играет не последнюю роль.

В мире насчитывается около 24 000 медицинских физиков \cite{IOMP}, из которых чуть более трети, или 8205, находятся в Соединенных Штатах \cite{AAPM1} и 2303 -- в Европе \cite{Lievens}. Дополнительная и более конкретная информация поступает от Американской ассоциации физиков в медицине (AAPM), которая ежегодно проводит опрос своих членов и предоставляет описательные статистические данные, имеющие отношение к национальной рабочей силе, но только на пространстве США. К примеру, у них на 2018 год, по данным 2565 респондентов, 51\% и 49\% имели степень магистра и доктора наук соответственно. Большинство (76\%)  медицинских физиков занимались радиационной онкологией в качестве своей основной специальности, и почти все (94\%) были заняты на полный рабочий день, и только 3\% были фрилансеры-консультанты. Как видно, проблем с численностью медицинской рабочей силы нет, в целом она находится в равновесии с потребностями страны. В США и нескольких других странах медицинская физика четко определенная, устоявшаяся и зрелая профессия. 

В СССР медицинская физика появилась в 60--e годы прошлого века, на данный момент в России насчитывается около 500 человек при очень низкой <<плотности>> населения \cite{Костылёв}, а в нашей стране и вовсе на 468 человек меньше \cite{Тарутин}. 


