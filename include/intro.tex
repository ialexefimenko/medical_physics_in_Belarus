
\section{Введение}{\parindent 1.27cm}

Почти сразу же после открытия рентгеновских лучей в конце XIX века \cite{W. C. Röntgen} ионизирующее излучение нашло применение в диагностике и лечении широкого спектра заболеваний человека, а также в промышленности, научных кругах, энергетике и национальной обороне. Несмотря на то, что мировое федеральное правительство содействовало созданию сообщества профессионалов, обеспечивающих безопасное и полезное использование радиации, число специалистов по радиации в последнее время тревожно сократилось,, когда в нашей стране еще происходит ее развитие, о чем свидетельствуют документы нескольких уважаемых организаций \cite{USGAO 2014}\cite{HPS 2013}\cite{NCRP 2015}. Совсем недавно МКРЗ опубликовали заявление под названием <<Где находятся специалисты по радиации?>>, которая предупредила о возможности того, что будущие национальные потребности в ряде соответствующих секторов, включая медицину, могут остаться неудовлетворенными \cite{NCRP 2015}.

Подчеркивается большая неопределенность, сильная зависимость от предполагаемых входных параметров и высокий уровень сложности проблемы в подходе к профессии медицинский физик \cite{Mills MD.2014}. Кроме того, она продолжает существенно развиваться. Разумно рассмотреть возможность того, что трудовых ресурсов может быть недостаточно, и продумать стратегии мониторинга риска для мирового сообщества медицинских физиков и, затрагивая работников лучевой терапии в здравоохранении Республики Беларусь, где профессия еще неустоявшаяся, но не с плохим успехом в решении поставленных задач. Несмотря на высокие различия в подготовке специалистов, проблема общая -- образовательный сектор с ее неминуемым упадком оценивания модели действительности. И, выходя на мировой уровень, в первую очередь необходимо разработать и внедрить приспособления и технологии для обеспечения качества диагностических и терапевтических процедур, организовать работу по разъяснению работникам организации здравоохранения вопросов обеспечения безопасности пациентов и работников, после этого и решение задач будет обсуждаться совместно с международным сообществом.

\addimg{ed}{1}{Пути обучения в области медицинской физики \cite{Silverstein}.}{ed}
